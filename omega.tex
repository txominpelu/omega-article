
# Introduction

## Difficulties (Problematica) with cluster scheduling

Scheduling is a problem that consists on planning the execution of a
set of computations that we'll call jobs in an execution environment
with a limited amount of resources. Initially scheduling was studied
in the context of operating systems however more recently with the
popularization (?) of datacenters a different range of scheduling
techniques are being applied for the allocation of resources in
clusters. 

Clusters can run both batch jobs and long running jobs (services), two
types of jobs that represent two completely different requirements.
The former requires most of the resources in a cluster however most of
the jobs executed in modern clusters (google, yahoo references) are
batch jobs with short timespan.

One can measure the good use of a cluster by considering if the
quality constraints required by services are satisfied, the execution
time of batch jobs and the percentage of cluster utilization. The
quality of the scheduling algorithm used for a cluster affects all of
this metrics so considering the costs involved in running a cluster in
a datacenter and the impact in bussiness terms of having the
appropriate quality of service for jobs is obvious the importance of
having a good scheduling algorithm. 


## Factors to consider when doing scheduling

- Independent frameworks with different scheduling needs (and
  currently different implementations  of the scheduling) (e.g Hadoop
  schedules differently from MPI)
- Those frameworks need to access the same data (they cannot run in
different clusters or otherwise they'll need to replicate the data)

So => Ideally there's a need to allow the different frameworks to have an
scheduling that is adapted to their specific needs while they need to
share the same cluster so they need to coordinate to be able to share
the available resources.

- Other needs:

 - Main use of a cluster: batch processes over distributed data
   (usually map-reduce but it can also be MPI) it is a really specific
   type of job that needs :
     - Data locality (so scheduling needs to be done so that jobs that
     process information that is stored in a specific node of the
     cluster can be executed in that node)
     - Many jobs, that don't take too long to execute
 - Needs to deal with services and jobs (two different needs)
 - Needs to scale when increasing the number of machines and the
   number of jobs in the cluster
 - Ideally the scheduler needs to be simple and should not need to be
   modified when new frameworks with different needs are added

## Solutions for coordinated cluster scheduling

### Centralized scheduler

Pro:

- Control over the whole cluster
- Simplest solution when there's only one framework running in the cluster

Problems:

- Complexity
- Can't adapt to new frameworks with new policies & needs
- Doesn't scale

### Two-Level scheduler



Glossary

- Frameworks (distributed application)
- Job
- Task
