
Difficulties (Problematica) with cluster scheduling


Scheduling is a problem that consists on planning the execution of a
set of computations that we'll call jobs in an execution environment
with a limited amount of resources. Initially scheduling was studied
in the context of operating systems however more recently with the
popularization (?) of datacenters a different range of scheduling
techniques are being applied for the allocation of resources in
clusters. 

Clusters can run both batch jobs and long running jobs (services), two
types of jobs that represent two completely different requirements.
The former requires most of the resources in a cluster however most of
the jobs executed in modern clusters (google, yahoo references) are
batch jobs with short timespan.

One can measure the good use of a cluster by considering if the
quality constraints required by services are satisfied, the execution
time of batch jobs and the percentage of cluster utilization. The
quality of the scheduling algorithm used for a cluster affects all of
this metrics so considering the costs involved in running a cluster in
a datacenter and the impact in bussiness terms of having the
appropriate quality of service for jobs is obvious the importance of
having a good scheduling algorithm. 




Glossary

- Job
- Task
