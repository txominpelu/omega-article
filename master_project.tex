\documentclass{svjour3}                     % onecolumn (standard format)
\usepackage[english, activeacute]{babel} %Definir idioma español
\usepackage[utf8]{inputenc} %Codificacion utf-8
\usepackage[autostyle]{csquotes}
\usepackage[backend=biber]{biblatex}

\begin{document}

\title{ Applying smart policies to dynamic resource managers }

%\titlerunning{Short form of title}        % if too long for running head

\author{ Inigo Mediavilla }
\institute{ UPMC Master STL \at
              \email{imediava@gmail.com}           %  \\
}

\date{\today}

\maketitle

\begin{abstract}
TODO
\end{abstract}

\section{Introduction}



\section{Priorities}

A really simple yet fairly powerful way to consider business needs when many frameworks
want to use the same resources at the same time is assigning priorities. In the competition
for resources, tasks with higher priorities will always win against lower priority tasks. In
the case where lower priority taks have already their resources assigned preemption will be
used to ensure that the order is respected and that low priority tasks don't slow down high
priority tasks even if this causes some overhead for the duplicated effort that lower priority
tasks need to do when they're preempted.

In practice this works well because by assigning the higher priority to the services, 
that provide the higher business value [ref], we can ensure their high availability since 
no other job can preempt them, and give them exactly the resources they need. Assigning the 
Lower priority levels to batch jobs works well because jobs are usually less important
than services but specially because they're more flexible in the allocation of their tasks,
and they deal well with preemption.


%% Idea:


%% Split into small improvements to the things that are not explained in the Omega paper
%% The simulator created can help prove the situation where some of the schedulers doesn't
%% work well and to test some of the improvements' performance.
%%
%% 1. Explaining how to deal with giving preference based on business value
%%    - Show in what situations just mediation for conflicts is not enough
%%       - When some jobs have more priority than others for any allocation (superjobs, e.g services)
%%           (Easy, higher priority jobs can preempt lower priority jobs but not
%%            the opposite - 
%%               -Means that higher priority jobs must behave with responsability 
%%               -Lower priority jobs need to allow preemption (or provide interruptions - can stand-by)
%%               - Can there be any low priority - non-preemptive jobs? why? How to
%%                 deal with that?
%%               - On preemption jobs need to be notified 

%%       - When some jobs have more priority than others for certain allocations (play with
%%         bidding to ensure fairness between same-level frameworks)
%%          - What to base bidding on?
%%             - Job almost finished
%%             - Many other tasks depend on one
%%             - Lower bidding for extending the resources dynamically
%%       - What jobs allow preemption and what jobs don't?
%%
%%   1.1. How can monitoring ensure a good use of the cluster and the fairness provided
%%        by policies?
%%       - Show on simulations of how good and bad fairness levels can be detected and
%%         what can cause a bad fairness level (e.g rogue frameworks, faster
%% 
%% 2. How Omega can be improved when two (or many) jobs want to work over the same data 
%%    (e.g in case of conflicts when do they retry, how to optimize that?)

\section{Experiments}
Two-level schedulers have been proven to deal badly with long scheduling decision times and
when resources are not freed frequently, providing a non-optimal utilization of the resources
of the cluster.

Omega doesn't have a clear policy to mediate between frameworks to provide coordination so 
that the execution can be improved.

Examples of situations where some global policies can improve the performance of the cluster:

 - A job takes most of the resources but it won't finish soon while other jobs would finish
   quickly but they are not given the opportunity to use the resources
 - A job that many other jobs depend on is ignored while other job whose execution is less
   relevant is executed first
 - A job that can run dynamically adjusting it's size takes resources of other job that needs
  a fixed amount

How many of those examples cannot be resolved with priorities? 


\section{Experiments}


\section{Bibliography}


\printbibliography


\end{document}
