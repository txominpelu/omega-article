\documentclass{svjour3}                     % onecolumn (standard format)
\usepackage[english, activeacute]{babel} %Definir idioma español
\usepackage[utf8]{inputenc} %Codificacion utf-8
\usepackage[autostyle]{csquotes}
\usepackage[backend=biber]{biblatex}

\addbibresource{omega-article.bib}
\addbibresource{other.bib}

\begin{document}

\title{ Applying smart policies to dynamic resource managers }

%\titlerunning{Short form of title}        % if too long for running head

\author{ Inigo Mediavilla }
\institute{ UPMC Master STL \at
              \email{imediava@gmail.com}           %  \\
}

\date{\today}

\maketitle

\begin{abstract}
TODO
\end{abstract}

\section{Table of Contents}

\tableofcontents

\section{Introduction}

Scheduling is planning the execution of a set of computations that
are called jobs in an execution environment with a limited amount of
resources.

If we consider the size of the clusters of the biggest companies
in the web, their resource consumption and the economical value of the
applications that run on them we can quickly realize of the
economical, ecological and business impact of a good utilization of
these clusters that depends significantly on the quality of
the schedulers that allocate their resources.

Besides, with the advention of the cloud platforms the ``prestation'' of clusters
has become a business in itself. Initially thought for hosting
webservers, databases and distributed services virtual machines
are finding other uses for their computational resources such as
distributed parallel processing that increases the possibility
of manipulate and make sense out of bigger amounts of data. This
different uses of cluster resources increases
even more the importance of the process of scheduling as well of the
complexity to handle all the different kinds of applications.

This relevance of the scheduling process constitutes a good
''incentivation'' for the scientific community to try to find the best
scheduling strategies to optimize the utilization of clusters. An
incentivation that can be facilitated by the public release of the
traces of the clusters of many big companies like [ref] that allow
anyone interested in testing improvements to the existing scheduling
models to try them against the real production traces of the clusters
of the biggest companies.

%% Difficulties

Scheduling has a great impact in the well functioning of a cluster in
terms of quality of service and running costs.  Unfortunately
designing a cluster scheduler is a difficult task. Many types
of applications can run in a cluster, with completely different
scheduling strategies that the scheduler needs to support while
remaining simple a fast enough to deal with the increasing size of
the clusters and the tighter scheduling deadlines imposed by recent
data processing frameworks.

Besides, the scheduler needs to able to deal with conflicts when many
applications try to use the same resource at the same time and manage
to offer every application a fair share of the resources of the cluster
according to a list of defined criteria.

Previous approaches to scheduling either have problems to deal with 
having to run a diversity of applications, cannot scale when a cluster
grows or don't achieve a high utilization of the resources. Two level 
schedulers like Mesos \cite{Hindman10mesos:a} do provide scalability 
and flexibility to run different applications but have some limitations when
trying to achieve a good utilization of the resources of the cluster
due to the pessimistic lock that is held with their offers and can reach
deadlocks when two applications do resource hoarding. Omega \cite{41684}
offers a more open model based on optimistic locking that avoids the 
inconveniences of the two level model. However only the model behind Omega
has been publicly exposed, so no available implementation exists that can
be used to test Omega in different contexts. Besides, the Omega paper
does not propose an explanation to achieve fairness in the context where
frameworks compete for the same resource or how they avoid problems with
``rogue'' frameworks apart from a basic notion of priority for the former 
and post-facto monitoring for the latter.


%% Be careful cause the simulator could be considered as such implementation

In section \nameref{approaches} I will describe the archetypical situation
that a cluster scheduler faces in a datacenter and I will show how the
existing approaches that have been proposed till the moment deal with it,
showing their negative and their positive aspects.




\section{Existing approaches}
\label{approaches}

\subsection{Solutions for coordinated cluster scheduling}

In the following section we will explain some of the existing models
of cluster schedulers considering their advantages and disadvantages
having into account the criteria introduced in the previous
section. Finally we will introduce the model Omega \cite{41684}
proposed by engineers at google that aims to overcome many of the
limitations of the other approaches.

\subsubsection{Centralized scheduler}

The first schedulers were designed following a monolithic architecture
with a central scheduler taking care of distributing the execution
of jobs across the nodes of the cluster. This design has, at least in
the beginning, the advantage of being simple since there's only one
scheduler assigning resources. The central scheduler controls the
state of the whole cluster, what makes it capable of taking really
good scheduling decisions specially when all jobs have the same
scheduling requirements. 

However this design that is simple and capable of taking good
decisions in the beginning, has many problems as the size of the
cluster increases or as the jobs executed in the cluster
diversify. When the number of jobs to execute increases, the central
scheduler needs to be able to keep with the pace of jobs that are
planned. On top of that a single scheduler represents a single point
of failure. Parallelization can atenuate both problems but just at the
expense of adding complexity and introducing the need for synchronization
between the schedulers.

Above all, for a centralized scheduler it is difficult to adapt
adequately to the heterogeinity of jobs that is executed in today's
clusters \cite{37201}. Trying to define different
scheduling paths for the different types of workloads is difficult
because it requires algorithms to identify the type of a job, and comes
at the expense of the scheduler's complexity. Some schedulers try to
support different policies by implement really complex algorithms
involving different weighting factors that provide rough
approximations of the objectives of every kind of workload. Others
like google's run different scheduling logic for every kind of
workload but it usually implies that users of the scheduler need to
have a really good understanding of the internals of the scheduling
algorithm to be able to optimize the execution of jobs. In both cases
the solution implies an additional layer of complexity to an already
difficult task.

Besides this model requires modifying the code of the scheduler every
time a new type of workload appears. 

Advantages:

\begin{itemize}
  \item Control over the whole cluster. The scheduler is capable of taking good scheduling decisions.
  \item Good cluster utilization
  \item Simplest solution when there's only one framework running in the cluster
\end{itemize}

Disadvantages:

\begin{itemize}
  \item Single point of failure
  \item Doesn't scale. The central scheduler is a bottleneck and
    slower jobs block other jobs that are faster to schedule.
  \item Can't adapt to new frameworks with new policies and needs
  \item Complexity
\end{itemize}

\subsection{Static partitioning}

To overcome the limitations that the previous model had when dealing
with clusters where the are many kinds of jobs with different scheduling
needs, static partitioning allows to split the resources of the cluster
in as many partitions as the kinds of jobs that we want to
treat independently. This strategy is attractive because it allows to
have an optimized dedicated scheduler for every kind while being
relatively simple since every scheduler manages its portion of the
cluster without any risk of conflicts with the others. It was
for a while the model chosen by companies like Yahoo.

Nonetheless this strategy's drawback is that since the different 
categories run in different, isolated portions of the cluster the
resources of the cluster are wasted since it is not possible for
a partition to use the resources that another partition is not using.

Advantages:

\begin{itemize}
    \item Simple
    \item Every framework can use their specific scheduler that is specialized
    \item There's no possibility of conflicts between frameworks when
      accessing the resources
\end{itemize}

Disadvantages:

\begin{itemize}
  \item Difficult to decide what is the right partition of the resources
between frameworks
  \item Non-optimal utilization of the resources of the cluster, for example
if one of the frameworks is idle other framework cannot use those
available resources since the repartition is done statically
\end{itemize}

\subsection{Two-Level scheduler}

Two level schedulers give applications, also called
frameworks, the ability to schedule their tasks independently while
having access to all the resources available in the cluster. This
is achieved with an architecture that is based in a master that manages
the cluster's resources and offers them to the frameworks who can accept 
the offers, notifying the master of the tasks they want to run and the
resources they want to run them on, or reject them.

The resources of the cluster are provided by the slaves that register to 
the master at startup providing information about the amount of memory, CPU
and network bandwidth  that they put at master's disposal.  

Frameworks can satisfy their needs (for example data locality \cite{chung_maximizing_2006} ) by 
rejecting the offers that don't interest them and accepting those who do, thus
ensuring per framework needs despite a centralized resource manager. 

When the interests of two frameworks conflict, for example two
frameworks want to run their jobs in the same slave because they both
access the same data, there are techniques to ensure that all
frameworks can have their fair share of the resources they need. In
the case of data processing jobs where tasks don't take long to
execute a technique called delay scheduling \cite{zaharia_delay_2010}
manages to provide good data locality for the tasks without delaying
the execution of the whole job. Nonetheless, when two competing
frameworks use resource hoarding to incrementally acquire resources
before they start a job deadlocks can emerge.

At the cluster level, global policies are enforced by the way that the
master distributes the offers to the frameworks. A cluster manager
like Mesos \cite{Hindman10mesos:a} provides two policies one based on
fairness \cite{AjtaiANRSW1998} and another on strict priorities,
custom policies can also be implemented. However if this policies are
too simple the central scheduler of Mesos can be extended with new
policies.

Fail-over on the master is provided with machines that run as backups of the master and
are capable to restore the master's state when it falls. Schedulers and slaves can find
the new master to register to with centralized configuration services like Zookeeper \cite{_apache_????}.

Partitioning of the resources of the cluster is provided thanks to Linux containers \cite{_linux_????}
an isolation technique that allows to run processes in an allocated subset of the 
resources of the machine without interferences from other processes. 

An optimization to the model of resource offers is the possibility for the frameworks
to define filters that describe the kinds of offers that a framework will not be interested
in, this way avoiding unnecessary exchanges between the slaves and the masters.

\subsubsection{Execution process}

A simple example where we have only one framework that wants to run a 
job formed by a single task will serve to explain all the steps that happen
before a computation can take place in the cluster:

\begin{enumerate}
\item Master starts. One of the candidates is selected as the master and 
the others remain as backup masters that will take the place of the current leader in
case it fails. The leader will start listening to the registration of slaves and 
frameworks.
\item The slaves start. They find the current master through a distributed configuration 
service and they register themselves by notifying the resources the put at the disposal
of the cluster.
\item The framework is launched. To execute its job the framework
  first registers to the master and starts listening to resource
  offers.
\item Since our framework is alone in the cluster, the master makes an offer with
all the resources available to our framework.
\item The framework receives the offer and accepts it. It does so by responding to the
offer with a list of the tasks that it wants to run specifying for every task: how to 
execute the task and what resources will be dedicated to the execution.
The response will only be valid if the resources assigned to all the tasks are a subset
of the resources offered by the master.
\item When the master receives the proposition to run the tasks. It
  allocates the requested resources inside containers in the slaves
  specified in the response and it launches the process to compute the
  task. To be able to run the tasks the slave obviously needs to have
  access to the binaries to run the process what is usually done
  either making the binaries accessible to the whole cluster through a
  distributed filesystem like the Google File System
  \cite{ghemawat_google_2003} or Amazon's S3 \cite{_aws_????} or by
  ensuring manually that the slaves have all the binaries in a public
  accessible folder.
\item Once the computation has finished the slave notifies the master that the resources
are available again so the master can continue offering them to the frameworks.
\end{enumerate}


%% (add diagram over how mesos works?)

Advantages:

\begin{itemize}
  \item
  Really simple central scheduler (master in Mesos terms) that has
  fail-over 
 \item
  Flexible: every framework decides how it schedules its tasks, the
  master only offers resources following fairness 
 \item
  Through offer rejection, frameworks can achieve high data locality
  only accepting offers that correspond to nodes where the data is
  stored. To avoid waiting indefinitely a technique called delayed
  scheduling \cite{zaharia_delay_2010} has proved to achieve high data locality without
  sacrificing job execution time benefiting from the fact that tasks in
  data processing frameworks like Hadoop or MPI are short so resources
  are freed frequently.
 \item Scalability
 \item Specialized for jobs with high chunk and where schedulers don't take
too long. With this kind of job, resources are freed frequently and
every framework gets its fair share of the cluster.
\end{itemize}

Disadvantages:

\begin{itemize}
 \item This kind of scheduling encourages frameworks to demand more
   resources than they actually need in preparation for everything
   they want to run. If a framework wants to run a list of tasks in
   many steps it will demand resources for the whole process even if
   many resources will be idle during most steps.
 \item Difficult to place picky jobs, for example to do gang scheduling
frameworks in Mesos need to do resource hoarding what can lead to
deadlocks
 \item When a framework is offered some resources it
holds the lock for the resource for the time the offer is
available. If a framework hoards the resources or takes a long
time to respond all other frameworks that are actually ready to accept
the offer and execute their jobs will have to wait for the first
framework to respond to the offer.

For example in clusters that run service jobs or jobs with complex
placement constraints that take long time to schedule the performance
of two-level schedulers suffers.

 \item Framework schedulers are written in an indirect style. They
  take actions only when the resources are offered. 
\end{itemize}


\section{Different Scheduling Needs}

Services
Map/Reduce First Generation
Map/Reduce New Generation (lower latency, higher scheduling needs but also higher
chances for fairness..?)


\section{Other responsabilities of the scheduler}

Abstraction for the execution of tasks in multiple machines (e.g API)
  - Run this in machine X
  - Run this in machine X with X CPUs and Y GB Ram
  - Communicate with my other task with id X (the cluster should figure out where the task is)
  - Autorecover if task fails (if failure is due to the tasks retry n times, if it is due
    to a machine find another machine for the task)


\section{Omega Limitations}

In its simplest version Omega was implemented without incremental resource allocation for 
tasks and without fine grained resource management just like the first version implemented
for Omega in [ref]. This version had limitations ... 

But even with incremental resource allocation and fine grained resource management Omega
has some limitations, specially as the cluster ``utilization'' grows and the possibility for 
conflicts increases. When dealing with conflicts between tasks there are some factors to consider:

 - (Task duration) If the tasks that conflict (or just the one that is running) take
   minutes to finish, seconds or milliseconds
 - If the tasks have different priority levels
 - If the tasks are allowed to be preempted or not (if they have high availability 
   requirements they can't *Note1*)
   



   
*Note1* : Would it be useful if tasks could be associated an acceptable downtime threshold?
*Note2* : When two services (higher priority) compete for overlapping resources what to do?
 - The first one takes it and the other one doesn't get anything
 - Sharing the cluster time (usually not satisfactory)
 - Raise an alarm (unexpected situation - impossible to resolve appropriately)
%% Can Omega be a general model that can be used to emulate other scheduling techniques (e.g
%% Mesos, Sparrow..)
 
When frameworks want to schedule tasks in the cluster they could either specify the required
machines or allow to fallback to the given amount of resources without considering specific
tasks. It's important that in the case where the cluster provides a fallback mechanism 
this should be as simple as possible (for example just taking any other free machine).




\subsection{Desired repartition properties}

Read from  paper [Beyond Dominant Resource Fairness]

- Incentivize sharing resources
- Pareto efficient
- Strategy-proof
- Envy-free


\section{Priorities}

A really simple yet fairly powerful way to consider business needs when many frameworks
want to use the same resources at the same time is assigning priorities. In the competition
for resources, tasks with higher priorities will always win against lower priority tasks. In
the case where lower priority taks have already their resources assigned preemption will be
used to ensure that the order is respected and that low priority tasks don't slow down high
priority tasks even if this causes some overhead for the duplicated effort that lower priority
tasks need to do when they're preempted.

In practice this works well because by assigning the higher priority to the services, 
that provide the higher business value [ref], we can ensure their high availability since 
no other job can preempt them, and give them exactly the resources they need. Assigning the 
Lower priority levels to batch jobs works well because jobs are usually less important
than services but specially because they're more flexible in the allocation of their tasks,
and they deal well with preemption.


%% Idea:


%% Split into small improvements to the things that are not explained in the Omega paper
%% The simulator created can help prove the situation where some of the schedulers doesn't
%% work well and to test some of the improvements' performance.
%%
%% 1. Explaining how to deal with giving preference based on business value
%%    - Show in what situations just mediation for conflicts is not enough
%%       - When some jobs have more priority than others for any allocation (superjobs, e.g services)
%%           (Easy, higher priority jobs can preempt lower priority jobs but not
%%            the opposite - 
%%               -Means that higher priority jobs must behave with responsability 
%%               -Lower priority jobs need to allow preemption (or provide interruptions - can stand-by)
%%               - Can there be any low priority - non-preemptive jobs? why? How to
%%                 deal with that?
%%               - On preemption jobs need to be notified 

%%       - When some jobs have more priority than others for certain allocations (play with
%%         bidding to ensure fairness between same-level frameworks)
%%          - What to base bidding on?
%%             - Job almost finished
%%             - Many other tasks depend on one
%%             - Lower bidding for extending the resources dynamically
%%       - What jobs allow preemption and what jobs don't?
%%
%%   1.1. How can monitoring ensure a good use of the cluster and the fairness provided
%%        by policies?
%%       - Show on simulations of how good and bad fairness levels can be detected and
%%         what can cause a bad fairness level (e.g rogue frameworks, faster
%% 
%% 2. How Omega can be improved when two (or many) jobs want to work over the same data 
%%    (e.g in case of conflicts when do they retry, how to optimize that?)

\section{Experiments}
Two-level schedulers have been proven to deal badly with long scheduling decision times and
when resources are not freed frequently, providing a non-optimal utilization of the resources
of the cluster.

Omega doesn't have a clear policy to mediate between frameworks to provide coordination so 
that the execution can be improved.

Examples of situations where some global policies can improve the performance of the cluster:

 - A job takes most of the resources but it won't finish soon while other jobs would finish
   quickly but they are not given the opportunity to use the resources
 - A job that many other jobs depend on is ignored while other job whose execution is less
   relevant is executed first
 - A job that can run dynamically adjusting it's size takes resources of other job that needs
  a fixed amount

How many of those examples cannot be resolved with priorities? 


\section{Experiments}

\section{Scheduling as an abstraction layer for distribution?}
%% Scheduling also consists in providing an abstraction layer for
%% computations to execute their jobs in a distributed environment
%% without having to manage the distribution of the computations
%% or the binaries to execute them.

%% Succintly explain the main problems of the main approaches
%% (forward reference) and the Omega alternative (forward reference)

%% Explain the limitations of Omega (for later)



\section{Bibliography}


\printbibliography


\end{document}
