\documentclass{svjour3}                     % onecolumn (standard format)
\usepackage[english, activeacute]{babel} %Definir idioma español
\usepackage[utf8]{inputenc} %Codificacion utf-8
\usepackage[autostyle]{csquotes}
\usepackage[backend=biber]{biblatex}

\begin{document}

\title{ Applying smart policies to dynamic resource managers }

%\titlerunning{Short form of title}        % if too long for running head

\author{ Inigo Mediavilla }
\institute{ UPMC Master STL \at
              \email{imediava@gmail.com}           %  \\
}

\date{\today}

\maketitle

\begin{abstract}
TODO
\end{abstract}

\section{Experiments}
Two-level schedulers have been proven to deal badly with long scheduling decision times and
when resources are not freed frequently, providing a non-optimal utilization of the resources
of the cluster.

Omega doesn't have a clear policy to mediate between frameworks to provide coordination so 
that the execution can be improved.

Examples of situations where some global policies can improve the performance of the cluster:

 - A job takes most of the resources but it won't finish soon while other jobs would finish
   quickly but they are not given the opportunity to use the resources
 - A job that many other jobs depend on is ignored while other job whose execution is less
   relevant is executed first
 - A job that can run dynamically adjusting it's size takes resources of other job that needs
  a fixed amount

How many of those examples cannot be resolved with priorities? 


\section{Experiments}


\section{Bibliography}


\printbibliography


\end{document}
