\documentclass{svjour3}                     % onecolumn (standard format)
\usepackage[english, activeacute]{babel} %Definir idioma español
\usepackage[utf8]{inputenc} %Codificacion utf-8
\usepackage[autostyle]{csquotes}
\usepackage[backend=biber]{biblatex}
\usepackage{graphicx}


\begin{document}

\title{ Improving dynamic allocation in cluster schedulers }

%\titlerunning{Short form of title}        % if too long for running head

\author{ Inigo Mediavilla }
\institute{ UPMC Master STL \at
              \email{imediava@gmail.com}           %  \\
}
\date{\today}
\maketitle

Scheduling is planning the execution of a set of computations that are
called jobs in an execution environment with a limited amount of
resources.

In the current context of todays' datacenters where clusters grow
really quickly reaching thousands of machines, and the number of jobs
increase at an even higher pace, schedulers need to be able to keep up with
these trends. This is even more difficult as the type of jobs that run
in the clusters diversifies and the scheduling constraints with it.

Previous models for doing scheduling were not ready to deal with all
these circumstances. The centralized model where all tasks are
scheduled by a centralized monolithic scheduler is not flexible enough
to adapt to different scheduling constraints and cannot keep up as the
size of the cluster or the number of jobs increase
significantly. Static partitioning allows frameworks to schedule
independently but it doesn't use the resources of the cluster
efficiently since resources that are not used in a partitioning cannot
be used by another. Fortunately, recently two dynamic allocation
models have been proposed that are more adapted to deal with the
circumstances that are present in todays' clusters.

The first to appear was the two level model represented by Mesos that
is based on a centralized scheduler (called master according to the
terminology used by Mesos) that has direct access to the state of the
cluster and knows what resources are available. At a lower level the
frameworks that want to run tasks in the cluster register to the
master who will make offers for the available resources. Frameworks
have the right to accept or refuse the offers they get. When they
accept they just need to specify what tasks they want to run and what
resources they want to dedicate to them.

The model behind mesos works well in general terms but it has some
inefficiencies and it doesn't work well with some specific cases. For
example the model is based on offering the available resources to one
framework at a time, this means that if a framework doesn't reply
quickly to the offer all the other frameworks have to wait. In the
case where many frameworks have registered but only one of them is
ready to launch a task if the proactive framework is unlucky he may
need to wait for all other frameworks to reply negatively before it
can receive its offer.

To overcome some of these limitations engineers at Google developed a
new model that they called Omega. Omega is a model that is based on
making the state of the cluster visible to all
frameworks. Frameworks make demands to the scheduler for resources
when they want. Since frameworks are kept up to date on the state of
the cluster most of the time they make demands for resources that
are free and their demands are accepted. However if two or more
frameworks make a request for the same resource at the same time the
central scheduler takes care of resolving the conflict and notifying
the frameworks whose offers haven't been accepted. This strategy is
based in what is called optimistic locking in opposition to the
pessimistic locking carried out by Mesos.

However, whereas Mesos is a mature opensourced project with many
interesting features, Omega has only been proposed in the paper
published by Schwarzkopf et al. and there is no mature implementation
of the scheduler publicly available. Besides, the paper where Omega is
described doesn't provide any information on the problem of how to
ensure that frameworks competing for resources get their fair share of
the cluster's resources, or how to solve the dilemma of how to provide
fairness in allocations while preserving datalocality a property that
ensures that jobs are executed next to the data they need.

This project presents a prototype implementation of both Mesos
and Omega. Then it fills the gap left by the paper that
describes Omega by explaining how to add two policies (priorities and
DRF) to ensure the consideration of business relevance with optimistic
locking. It also describes how we to achieve both
datalocality and fairness with DRF in Omega thanks to delay scheduling
a technique where frameworks wait for a timeout before they accept
offers for nodes that don't contain the data they require. The
last contribution is a description of how to emulate optimistic
locking with a two level scheduler, an interesting technique that
opens the door to setup mixed workloads in two level schedulers.

However the implementations that this project has made of Mesos and
Omega are really simple prototypes that make many simplifications when
compared with the actual challenges faced by real world cluster
schedulers and are of use only for experimentation and to facilitate
the comprehension of the model. 

Therefore further work in this project would include as a first step
improving the stability of the implementations to be able to do test
on them with some of the traces of real clusters that have been
publicly released, this way being able to test in a near real
environment the modifications that have been proposed in the project.


%% La fiche Dr synthèse doit contenir les éléments suivants :

%% 1- Contexte général du stage
%% Contexte et bref état de l'art : travaux dans ce domaine

%% 2- Problème étudié
%% Présentation du problème. Importance de la question, enjeux. Travaux connexes.

%% 3- Votre contribution
%% Choix et solutions que vous proposez (pas de technique, seulement les idées et les grandes lignes !)

%% 4- Analyse de votre travail
%% Apport de votre travail, du point de vue théorique, expérimental. Qualités et limitations de votre solution.

%% 5- Bilan et perspectives
%% Intérêt de votre approche, apport de votre contribution au domaine; quelle est la suite à donner à votre travail.


\end{document}
